\section{Introducción}

El flujo de calor se define como la variación de calor por unidad de tiempo. La ley de Fourier relaciona el flujo de calor a través de un material con el área a través de la cual fluye el calor, la distancia a través de la cual fluye y la diferencia de temperaturas entre las caras del material. La relación está dada por una constante de proporcionalidad $\kappa$ llamada \emph{conductividad térmica}.

El propósito del presente es determinar esta constante para diferentes muestras. Para esto, se impone una diferencia de temperaturas en las caras opuestas de cada muestra usando una cámara conectada a una fuente de vapor y un bloque de hielo. Esta constante queda determinada si se conoce la rata a la cual se funde el hielo y la geometría de la muestra.