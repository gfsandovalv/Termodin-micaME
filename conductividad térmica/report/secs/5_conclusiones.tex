\section{Conclusiones}

En general, se concluye que los datos no proporcionan una medición exacta de la conductividad, pues se observan errores mayores al $10\%$. En particular, para el yeso y el MDF, deben ser descartados directamente por su nivel de error en esta toma de datos. 

Posibles fuentes de error:

\begin{itemize}
    \item Dificultad para determinar el área efectiva de conducción, la cual correspondía al área transversal del hielo. Este no se fundía de forma uniforme, sino que se deformaba su sección transversal a una elipse. El método que se usó fue medir directamente el bloque de hielo con un calibrador, sin embargo, el calibrador fundía el hielo donde hacen contacto al ejercer una ligera presión con la intensión de obtener la medida más certera.
    \item  El volumen de hielo fundido en una unidad de tiempo era mucho menor que las marcas del vaso de medición que se usó. Esto implicó que se tuviesen que medir intervalos largos de tiempo. Y dado el número de muestras a caracterizar, en general significó pocos puntos muestrales para las muestras, y tan solo una toma de datos. 
    \item Los intervalos largos también podrían implicar una complicación en cuanto a la determinación del área efectiva de conducción, pues sobre la muestra el agua se acumulaba en cierta cantidad al rededor del bloque de hielo, esto podría significar una mayor área efectiva que no fue tenida en cuenta ni en la medición de cada muestra ni en la caracterización del transporte de calor con el medio ambiente.
\end{itemize}